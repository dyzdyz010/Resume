% LaTeX file for resume 
% This file uses the resume document class (res.cls)

\documentclass{res}

\usepackage{ctex}
\usepackage{indentfirst}
\usepackage[colorlinks=true]{hyperref}

%\usepackage{helvetica} % uses helvetica postscript font (download helvetica.sty)
% \usepackage{newcent}   % uses new century schoolbook postscript font 
\setlength{\textheight}{9.5in} % increase text height to fit on 1-page 

\begin{document} 

\name{杜\ 艺卓\\[12pt]}     % the \\[12pt] adds a blank
				        % line after name      

\address{
  \bf  现居地址\\
    北京交通大学\\
    北京市海淀区,100044\\
    \href{mailto:du.yizhuo@icloud.com}{du.yizhuo@icloud.com}
}

\address{
  \bf 在线资料 \\
    \href{http://blog.duyizhuo.com}             {Blog}  \\
    \href{https://github.com/dyzdyz010}         {Github} \\  
    \href{http://segmentfault.com/u/dyzdyz010}  {SegmentFault}
}
                                  
\begin{resume}

\section{求职意向}
\noindent
    实习\\
    iOS 程序开发\\
    Cocos2d 游戏开发\\
    Node.js、Golang等服务器端开发\\

\section{教育经历}
\noindent
    北京交通大学\\
    2010-2014\\
    软件工程专业,数字媒体方向\\


\section{项目经历}
    \vspace{-0.1in}

    \begin{tabbing}
      \hspace{2.3in}\= \hspace{2.6in}\= \kill % set up two tab positions
      {\bf 交大资讯通(未上架)} \>  iOS应用  \>2011.秋 - 2012.春\\
                          \centerline{北京交通大学『新软攀峰』软件设计大赛二等奖}
    \end{tabbing}\vspace{-20pt}      % suppress blank line after tabbing
    收集交大周边信息并允许用户评论与分享。\\
    本人负责利用 {\bf Core Data} 与 {\bf ASIHttpRequest} 网络库实现客户端与服务器的数据交互与存储。

    \begin{tabbing}
      \hspace{2.3in}\= \hspace{2.6in}\= \kill % set up two tab positions
      {\bf iCollege} \>  PHP 网站  \> ~~~~~~~~~~~~ 2012.夏\\
                          \>『数据结构』课程项目
    \end{tabbing}\vspace{-20pt}      % suppress blank line after tabbing
    交大资讯通 iPhone 客户端的 Web 扩展,增加了 UGC 功能,并结合了 Google Maps 的使用。\\
    本人负责生成地标信息页面和地标信息展示页面的设计与制作,以及 Google Maps 接口的使用。

    \begin{tabbing}
      \hspace{2.3in}\= \hspace{2.6in}\= \kill % set up two tab positions
      {\bf PP Piano} \> Windows RT 应用 \> ~~~~~~~~~~~~ 2012.冬\\
                          \>微软『编程马拉松』二等奖
    \end{tabbing}\vspace{-20pt}
    Windows RT 钢琴模拟应用。\\
    本人负责音色文件读取、存储以及使用。

    \begin{tabbing}
      \hspace{2.3in}\= \hspace{2.6in}\= \kill % set up two tab positions
      {\bf 星际2争霸秘笈}  \>Windows RT 应用\> ~~~~~~~~~~~~ 2012.冬\\
                          \>微软『编程马拉松』特等奖
    \end{tabbing}\vspace{-20pt}
    Windows RT 『星际争霸2』展示应用。内含游戏宣传视频、三大种族基本信息、兵种详细信息、以及各族常用战术介绍。\\
    本人负责页面逻辑的架构,搜集图片资源和信息。

    \begin{tabbing}
      \hspace{2.3in}\= \hspace{2.6in}\= \kill % set up two tab positions          
      {\bf 鹿王本生(未上架)}  \> iPad 应用 \> ~~~~~~~~~~~~ 2012.冬\\
                          \centerline{2012年『北京市大学生动漫设计竞赛』一等奖(数字艺术类)}
    \end{tabbing}\vspace{-20pt}
    通过虚拟成像展示『鹿王本生』的故事。  \\
    本人负责利用 Vuforia SDK 生成指定图片的识别特征,并在程序中检测摄像头中的图像,进行匹配。

    \begin{tabbing}
      \hspace{2.3in}\= \hspace{2.6in}\= \kill % set up two tab positions
      {\bf 旅行心愿单(未上架)} \>  iOS 应用  \> ~~~~~~~~~~~~ 2013.春\\
                          \centerline{北京交通大学『新软攀峰』软件设计大赛一等奖}
    \end{tabbing}\vspace{-20pt}      % suppress blank line after tabbing
    允许用户拍照,给照片贴上标签,并分享给其他人。能够记录用户拍照的地点,并在地图中展示出来。
    本人负责制定网络接口规范并实现。

    \begin{tabbing}
      \hspace{2.3in}\= \hspace{2.6in}\= \kill % set up two tab positions
      {\bf 水果嘉年华(未上架)} \>  iOS 游戏  \> 2012.秋 - 2013.夏\\
                          \centerline{同『北京服装学院』合作项目}
    \end{tabbing}\vspace{-20pt}      % suppress blank line after tabbing
    与『北京服装学院』合作,基于 Cocos2d 的 iPhone 游戏。以各种水果为主题,由数十个小游戏组成,包含成就系统,并且可以将成绩分享到新浪微博。\\
    本人负责游戏暂停与游戏结束逻辑的构建与实现,并负责其中四个小游戏的全部实现。

    \begin{tabbing}
      \hspace{2.3in}\= \hspace{2.6in}\= \kill % set up two tab positions
      {\bf Moonlightter} \>   Golang 网站  \> ~~~~~~~~~~~~ 2013.春\\
                          \>课外 Golang 实践项目
    \end{tabbing}\vspace{-20pt}      % suppress blank line after tabbing
    http://moonlightter.duyizhuo.com\\
    允许用户选择并保存自己喜欢的颜色,以预览+十六进制代码的形式展现。\\
    前端使用了『Pure』CSS 框架,服务器端使用了 『Beego』Web 框架,数据库使用 『MongoDB』。

    \begin{tabbing}
      \hspace{2.3in}\= \hspace{2.6in}\= \kill % set up two tab positions
      {\bf Viki智能外教(已上架)} \>  iOS 应用  \> ~~~~~~~~~~~~ 2013.夏\\
                          \>业余外包项目
    \end{tabbing}\vspace{-20pt}      % suppress blank line after tabbing
    可以选择人工智能外教与其聊天,支持语音读写功能,使用了 『Google Tranlate』 与『百度翻译』等外部接口。\\
    本人负责程序数据层以及网络接口的构建与实现。

    \begin{tabbing}
      \hspace{2.3in}\= \hspace{2.6in}\= \kill % set up two tab positions
      {\bf StarTD(星际塔防)(未上架)} \>  iOS 游戏  \> ~~~~~~~~~~~~ 2013.夏\\
                          \>暑期实训项目
    \end{tabbing}\vspace{-20pt}      % suppress blank line after tabbing
    iOS 塔防游戏,实现了塔防的出现敌人,炮塔攻击等基本逻辑。使用了 『Ninevehgl』 游戏引擎和 『OpenGL』 结合,同时结合了 『UIKit』。\\
    本人负责游戏基础逻辑,地图文件的读取与解析,3D 场景解析等。

\section{计算机技能}
\noindent
    熟悉 {\bf iOS} 下 {\bf Foundation} 框架,熟悉 {\bf ASIHTTPRequest},{\bf AFNetworking},{\bf GCDAsyncSocket} 等第三方类库;\\
    熟悉 {\bf Cocos2d for iPhone} 游戏开发框架;\\
    熟练使用 {\bf iOS} 下 单例、工厂等设计模式,熟练使用 {\bf Delegate} 编程思想;\\
    了解 {\bf Node.js} 设计思想,了解 {\bf Express} 开发框架,有简单实践经验;\\
    熟悉 {\bf Golang} 语言特性,熟悉 {\bf Beego} 开发框架,有简单实践经验;\\
    了解 {\bf HTML,CSS,Javascript} 等前端开发语言,熟悉 {\bf Bootstrap,Pure} 等 CSS 框架,熟悉 {\bf Backbone.js,Angular.js} 等MV*框架;\\
    {\bf Mac} 重度用户,熟悉 {\bf Linux} 的日常命令及使用,熟悉 Mac 环境下的软件开发与部署;\\
    熟练使用 {\bf Git} 版本控制工具。
 
\section{曾获荣誉}   
\noindent       
    2012 - 2013 『大学生创新实验项目』国家级『优』,题目——『基于 AR 的空间物体解析与实现』。
 
\section{业余活动}
\noindent
    热爱新技术,关注技术前沿,《黑客与画家》、《MackTalk·人生元编程》读者;\\
    『星际争霸2』爱好者。
 
\end{resume}
\end{document}